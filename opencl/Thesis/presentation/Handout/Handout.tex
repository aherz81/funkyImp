\documentclass{article}
\usepackage[utf8]{inputenc}
\usepackage[ngerman]{babel}
\usepackage{fancyhdr}
\usepackage{listings}
\usepackage[a4paper,text={160mm,255mm},centering,headsep=5mm,footskip=10mm]{geometry}
\usepackage{nameref}
\usepackage{listings}
\usepackage{amsmath}
\usepackage{amssymb}
\usepackage{graphicx}

\usepackage{pgf}
\usepackage{tikz}
\pagestyle{fancy}



%Infos zum Arbeitsblatt
\lhead{Handout}
\chead{ }
\rhead{Begriffe}
\lfoot{Alexander Pöppl}
\rfoot{5. September 2014}
\begin{document}

\begin{itemize}
	\item \textbf{GPGPU Computing} - General Purpose GPU Computing
\end{itemize}

\section{OpenCL}

\begin{itemize}
	\item Offener Standard zur Durchführung von parallelen Berechnungen auf heterogenen Systemen
	\item Unterstützt GPUs, CPUs, Accelerators
	\item Implementierungen u.a. von Intel, AMD, NVidia, Apple
\end{itemize}

\section{OpenCL - Ausführungsmodell}

\begin{itemize}
	\item \textbf{Platform:} ``Konventionelle'' Ausführungsumgebung. Kann Devices nutzen, um Berechnungen durchzuführen.
	\item \textbf{Device:} Gerät, auf dem OpenCL-Code ausgeführt wird
	\item \textbf{Kernel:} Die Funktion, die auf der GPU ausgeführt wird. Muss kompiliert werden.
	\item \textbf{Memory:} Die Daten, die der Kernel nutzt. Müssen transferiert werden.
	\item \textbf{Work-Item:} Einzelner Thread, ein Durchlauf des Kernels
	\item \textbf{Work-Group:} Gruppe von Work-Items, die nebenläufig ausgeführt wird
\end{itemize}

\subsection{Einschränkungen}
\begin{itemize}
	\item Es werden immer nur Work-Items einer Work-Group gleichzeitig ausgeführt
	\item Vor Beginn der Ausführung einer neuen Work-Group muss die vorhergehende abgeschlossen sein
	\item $n_{Work-Items} \operatorname{mod} s_{Work-Group} = 0$	
\end{itemize}

\section{OpenCL - Speichermodell}
Siehe Graphik. \\

\begin{center}
	\includegraphics[width=0.75\linewidth]{../../images/memory_hierarchy}
\end{center}

\section{funkyIMP}
\paragraph{Domain Iterations} Intuition: \texttt{map} Operation auf mehrdimensionalen Arrays. \\

\begin{gather*}
		M = \begin{pmatrix}
			1 & 2 & 3 \\
			5 & 6 & 7  \\
			42 & 7 & 10
		\end{pmatrix} \\
		M' = M.\backslash(a,b) \{M[a,b] * 2\} \\
		\rightarrow M' = \begin{pmatrix}
			2 & 4 & 6 \\
			10 & 12 & 14  \\
			84 & 14 & 20
		\end{pmatrix}
\end{gather*}

\section{Statistik}
Sei $V$ eine Sammlung von Werten $v \in \mathbb{R}$. Dann sind das arithmetische Mittel \emph{avg}, die Standardabweichung \emph{stdDev} und der Standardfehler \emph{stdErr} definiert als:
\begin{gather}
\label{eq:suite_avg}
\operatorname{avg}(V) = \frac{\sum_{v \in V} v}{|V|}\\[2ex]
\label{eq:suite_stdDev}
\operatorname{stdDev}(V) = \sqrt{\frac{\sum_{v\in V} (v - \operatorname{avg}(V))^2}{|V|}}\\[2ex]
\label{eq:suite_stdErr}
\operatorname{stdErr}(V) = \frac{\operatorname{stdDev}(V)}{\sqrt{|V|}}
\end{gather}

\end{document}